\subsubsection{Description}
%Describe your concept of the pre-charge circuitry.
Pre-charge circuit is controlled by ECUA. (Isolation on PCB is refered from 8.3Electronic control unit acp (ECUA)) Our pre-charge is assembled from 3 resistors. 2 of them have the same value 470R, the third is mainly part of safety precaution and has value 2k2. There is always a chance that the resistor fails, in that case the remaining resistors may be used as fully functional pre-charge, only slower.

TS ON button sets whole car in state ‘pre-charging’. ECUA starts pre-charge process. It takes safety precautions and then relay and AIR are switched to close pre-charging current. Fully closed SDC and ECUA decision are needed to start pre-charging. If every point is fulfilled and pre-charge is successfully finished, second AIR closes. Pre-charge relay opens and car leaves pre-charge state. (Pre-charge relay is normally open type.)
% chybi schema - Honza doplni

\paragraph{Pre-charge safety on ECUA}
Except from what rules require we implemented several safety precautions. In case, that SDC error doesn’t occur and driver pushed TS ON button following safety precautions are taken to prevent switching AIR in case of problem that has not yet been detected.

First one is completely non-programmable protection against switching voltage difference by AIR. It uses voltage measurement and then comparators and logic to disable microcontroller decision in case of SW error.
Second one is measuring all states and voltages by microcontroller on ECUA which can determinate error before non-programmable protection would have to act.

Third (time-out) protection is used when everything seems to be OK, but the charging is too slow – caused by too high pre-charge resistance (any of pre-charge resistors fails), or some leakage of charge in capacitor or any other possible error occurs. If voltage difference is not equaled in time less them 2seconds, the ECUA stops pre-charge and waits 5 seconds before trying again. (In order to not overpower resistors.) If number of attempts to pre-charge is in this state higher then 8, something is clearly wrong and ECUA opens SDC and indicates error. Sending message about error and sets car into not-ready state.

\subsubsection{Wiring, cables, current calculations, connectors}
Describe wiring, show schematics, describe connectors and cables used and show useful data regarding the wiring.
\begin{itemize}
\item Give a plot “Percentage of Maximum Voltage” vs. time
\item Give a plot Current vs. time 
\item For each plot, give the basic formula describing the plots
\end{itemize}

Additionally, fill out the tables:

\begin{table}[H]
	\centering
	\caption{General data of the pre-charge resistor}
	\begin{tabularx}{\textwidth}{|X|X|}
		\hline
		Resistor Type: & \\[\TableSize]
		\hline
		Resistance: & \\[\TableSize]
		\hline
		Continuous power rating: & \\[\TableSize]
		\hline
		Overload power rating (1 sec): &  \\[\TableSize]
		\hline
		Overload power rating (5 sec): &  \\[\TableSize]
		\hline
		Overload power rating (15 sec): &  \\[\TableSize]
		\hline
		Voltage rating: & \\[\TableSize]
		\hline
		Cross-sectional area of the wire used: & \\[\TableSize]
		\hline
	\end{tabularx}%
	\label{tab:precharge-general}%
\end{table}%

\begin{table}[H]
	\centering
	\caption{General data of the pre-charge relay}
	\begin{tabularx}{\textwidth}{|X|X|}
		\hline
		Relay Type: & \\[\TableSize]
		\hline
		Contact arrangment: &  \\[\TableSize]
		\hline
		Continuous DC current:  & \\[\TableSize]
		\hline
		Voltage rating  & \\[\TableSize]
		\hline
		Nominal Coil Voltage: &  \\[\TableSize]
		\hline
		FET type: &  \\[\TableSize]
		\hline
		Maximum Drain-Source Current: &  \\[\TableSize]
		\hline
		Drain-Source Breakdown Voltage: &  \\[\TableSize]
		\hline
		On Charasteristics Gate Threshold Voltage: & \\[\TableSize]
		\hline
		Cross-sectional area of the wire used: & \\[\TableSize]
		\hline
	\end{tabularx}%
	\label{tab:precharge-relay}%
\end{table}%

\subsubsection{Position in car}
Provide CAD-renderings showing all relevant parts. Mark the parts in the rendering, if necessary.